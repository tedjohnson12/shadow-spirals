% Define document class
\documentclass[twocolumn]{aastex631}
\usepackage{showyourwork}

% Begin!
\begin{document}

% Title
\title{Spirals launched by a planet in a shadowed disk}

% Author list
\author{Ted Johnson}

\begin{abstract}
    The presence of a planet in a protoplanetary disk is known to launch pressure and
    density spirals in the disk. A disk that is shadowed (via an optically thick inclined inner disk)
    will change the shape of these spirals.
\end{abstract}

\section{Introduction}
\label{sec:intro}

A planet embedded in a protoplanetary disk will launch a spiral similar to those seen in galaxies.
Sound waves excited by the planet travel in the rotating disk at the local sound speed, but also follow
the Keplerian rotation of the disk material. This creates a spiral shape, as waves launched interior to
the planet outpace it and waves launched outside lag behind.

Most simulations of protoplanetary disks assume that the sound speed is constant throughout the disk,
however this may not always be the case. An interior, truncated disk that is misaligned to the main disk can
cast shadows on its surface, locally lowering the temperature and sound speed. Shadows can have two effects:
\begin{enumerate}
    \item Deflecting waves launched by a planet (similar to Snell's Law).
    \item Launching its own waves because the disk now has a pressure gradient in the $\phi$ direction.
\end{enumerate}

Both effects will affect an observation of a planet in a shadowed disk. In this paper we will simulate
such a shadowed disk, subtract the perturbations due to the shadow, and compare the spiral to the
shadowless case.

\section{Methods}
\label{sec:methods}
\subsection{Simulation setup}
\label{subsec:setup}
We use the magneto-hydrodynamics code Athena++\footnote{\url{https://github.com/PrincetonUniversity/athena}{https://github.com/PrincetonUniversity/athena}} 
\citep{stone2020} version 21.0 \citep{athena++developmentteam2021} to simulate a 2D gas disk. The simulation was set up in cylindrical coordinates with 149
cells in the $\ln{r}$ direction with $r \in [0.4, 2.5]$ and 512 cells in the $\phi \in [-\pi, \pi]$ direction. The inner and outer boundaries have a user-defined fixed boundary
condition and the azimuthal boundary has a periodic boundary condition to ensure material can orbit the central star indefinitly. The dimensionality of the problem
is set up so that all masses are in $M_\odot$, distances are in AU, and times are in year/$2\pi$. This means that a planet placed in a circular orbit at $r=1$
will have an orbital period of $2\pi$.

The disk is set up so that $\rho = 1$ at $r=1$ and has the radial profile $\rho \propto r^{-1}$. The gas in the disk initially moves at the local keplerian velocity.
A shadow on the disk is prescribed by setting the sound speed -- analogous to setting the temperature as 
\begin{equation}
    c_s = \frac{\partial P}{\partial \rho} \propto T
\end{equation}
We ensure that the shadow does not move with the gas in its orbit by setting the cooling timescale to $10^{-6} \ll 1$. The sounds speed is essentially reset each time
the grid is updated.

The shadow is parameterized by three numbers: \texttt{h\_reduced\_factor}, \texttt{half\_shadow}, and \texttt{half\_unshadow},
which we will write here as $h$, $a$, and $b$, respectively. The former is the dimensionless magnitude of the sound speed reduction
and the latter two are the half-width of the regions with constant sound speed (i.e. no the transition region) and have
units of radians.

The sound speed outside the shadowed region can be computed
\begin{equation}
    c_{\rm un} = c_0 r^{-1}
\end{equation}

where $c_0 = 0.1$ is the sound speed at $r=1$. The effect due to the shadow is parameterized by the equation
\begin{equation}
    c_{\rm sh} = c_{\rm un} H(\phi)
\end{equation}

where 

\begin{equation*}
    \label{eq:H}
    \begin{array}{cccc}
        H = & h & \text{if} & (|\phi|~\%~\pi) \le a  \\
            & 1 & \text{if} & (\left | |\phi| - \frac{\pi}{2}\right |~\%~\pi) \le b \\
            & h + (1-h) \sin{\left ( \frac{\pi}{2}  \frac{|\phi| - b}{\frac{\pi}{2} - a - b} \right )} & & \text{otherwise}
    \end{array}
\end{equation*}


\begin{figure}
    % \includegraphics[width=0.5\textwidth]{}
    \caption{Initial gas temperature maps for our three setups. {\bf Left:} In the case where there is no shadow the temperature is azimuthally symmetric.
    {\bf Center:} In our ``narrow'' case we see a deep shadow covers a narrow region of the disk. This is analogous to a thin, optically thick inner disk.
    {\bf Right:} In our ``wide'' case we see a much shallower shadow that covers a significant portion of the disk. In this case $a,b=0$ and the third line of
    Equation \ref{eq:H} becomes a sine function for all $\phi$.
    }
\end{figure}

\begin{table}
    \begin{tabular}{cccc}
        \hline
        \hline
        Name & Width & $c_{sh}/c_{un}$ & $m_p$ \\
        \texttt{no\_shadow} & None & 0 & $10^{-4}$ \\
        
        \hline


    \end{tabular}
\end{table}




\bibliography{syw,spirals}

\end{document}
