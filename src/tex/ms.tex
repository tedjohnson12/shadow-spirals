% Define document class
\documentclass[twocolumn]{aastex631}
\usepackage{showyourwork}

% Begin!
\begin{document}

% Title
\title{Spirals launched by a planet in a shadowed disk}

% Author list
\author{Ted Johnson}

\begin{abstract}
    The presence of a planet in a protoplanetary disk is known to launch pressure and
    density spirals in the disk. A disk that is shadowed (via an optically thick inclined inner disk)
    will change the shape of these spirals.
\end{abstract}

\section{Introduction}
\label{sec:intro}

A planet embedded in a protoplanetary disk will launch a spiral similar to those seen in galaxies.
Sound waves excited by the planet travel in the rotating disk at the local sound speed, but also follow
the Keplerian rotation of the disk material. This creates a spiral shape, as waves launched interior to
the planet outpace it and waves launched outside lag behind.

Most simulations of protoplanetary disks assume that the sound speed is constant throughout the disk,
however this may not always be the case. An interior, truncated disk that is misaligned to the main disk can
cast shadows on its surface, locally lowering the temperature and sound speed. Shadows can have two effects:
\begin{enumerate}
    \item Deflecting waves launched by a planet (similar to Snell's Law).
    \item Launching its own waves because the disk now has a pressure gradient in the $\phi$ direction.
\end{enumerate}

Both effects will affect an observation of a planet in a shadowed disk. In this paper we will simulate
such a shadowed disk, subtract the perturbations due to the shadow, and compare the spiral to the
shadowless case.

\section{Methods}
\label{sec:methods}
\subsection{Simulation setup}
\label{subsec:setup}
We use the magneto-hydrodynamics code Athena++\footnote{\url{https://github.com/PrincetonUniversity/athena}} 
\citep{stone2020} version 21.0 \citep{athena++developmentteam2021} to simulate a 2D gas disk. The simulation was set up in cylindrical coordinates with 149
cells in the $\ln{r}$ direction with $r \in [0.4, 2.5]$ and 512 cells in the $\phi \in [-\pi, \pi]$ direction. The inner and outer boundaries have a user-defined fixed boundary
condition and the azimuthal boundary has a periodic boundary condition to ensure material can orbit the central star indefinitly. The dimensionality of the problem
is set up so that all masses are in $M_\odot$, distances are in AU, and times are in year/$2\pi$. This means that a planet placed in a circular orbit at $r=1$
will have an orbital period of $2\pi$. We chose a planet mass of $10^{-4}$ in order to create a detectable effect in the presence of noise from the shadow and
to not stray too far from the linear spiral regime.

The disk is set up so that $\rho = 1$ at $r=1$ and has the radial profile $\rho \propto r^{-1}$. The gas in the disk initially moves at the local keplerian velocity.
A shadow on the disk is prescribed by setting the sound speed -- analogous to setting the temperature as 
\begin{equation}
    c_s = \frac{\partial P}{\partial \rho} \propto T
\end{equation}
We ensure that the shadow does not move with the gas in its orbit by setting the cooling timescale to $10^{-6} \ll 1$. Essentially, the temperature is reset each time
the grid is updated.

The shadow is parameterized by three numbers: \texttt{h\_reduced\_factor}, \texttt{half\_shadow}, and \texttt{half\_unshadow},
which we will write here as $h$, $a$, and $b$, respectively. The former is the dimensionless magnitude of the sound speed reduction
and the latter two are the half-width of the regions with constant sound speed (i.e. not the transition region) and have
units of radians.

The sound speed outside the shadowed region can be computed
\begin{equation}
    c_{\rm un} = c_0 r^{-1}
\end{equation}

where $c_0 = 0.1$ is the sound speed at $r=1$. The effect due to the shadow is parameterized by the equation
\begin{equation}
    c_{\rm sh} = c_{\rm un} H(\phi)
\end{equation}

where 

\begin{equation*}
    \label{eq:H}
    \begin{array}{cccc}
        H = & h & \text{if} & |\sin{\phi}| \le \sin{a}  \\
            & 1 & \text{if} & |\cos{\phi}| \le \sin{b} \\
            & h + \frac{1-h}{2} \left (  1-\cos{\left ( \frac{2 \pi | \phi + m\pi |}{\pi - 2a - 2b}  \right )  } \right ) & & \text{otherwise}
    \end{array}
\end{equation*}
and
\begin{equation}
    \begin{array}{cccc}
        m = & 1 & \text{if} & \phi < -\frac{\pi}{2} \\
            & -1 & \text{if} & \phi > \frac{\pi}{2} \\
            & 0 & & \text{otherwise}
    \end{array}
\end{equation}
Figure \ref{fig:cs} shows the sound speed profiles used in this work.

\begin{figure}
    \begin{center}
        \includegraphics[width=3.5in]{figures/sound_speed.pdf}
        \caption{Sound speed as a function of azimuthal angle. Here we show how the sound speed is prescribed in our two science cases. The cases
        share a minimum and maximum sound speed, but the transitions between these values are very different.}    
    \end{center}
    \script{plot_sound_speed.py}
    \label{fig:cs}
\end{figure}


\begin{figure*}
    \includegraphics[width=1.0\textwidth]{figures/initial_temperature.pdf}
    \caption{Initial gas temperature maps for our three setups. Note that the radial axis is scaled logarithmically. {\bf Left:} In the case where there is no shadow the temperature is azimuthally symmetric.
    {\bf Center:} In our ``narrow'' case we see a near-constant shadow covers a narrow region of the disk. This is analogous to a thin, optically thick inner disk.
    {\bf Right:} In our ``wide'' case we see a much shallower shadow that covers a significant portion of the disk. In this case $a,b=0$ and the third line of
    Equation \ref{eq:H} becomes a sine function for all $\phi$.
    }
    \label{fig:setup}
    \script{plot_initial.py}
\end{figure*}

We will investigate two scenarios:
\begin{enumerate}
    \item A narrow shadow, analogous to that produced by a thin inner disk with a scale height of $\frac{h}{r} = 0.05$. We chose a value of $a$ based on the assumption that the inner disk
        is optically thick up to 3 scale heights. We call this case ``narrow''.
    \item A wide shadow, like that produced by a diffuse inner disk with a very high scale height. In this limit the temperature is simply a sinusoidal function of $\phi$. We call this case ``wide''.
\end{enumerate}
In order to adequately investigate these scenarios we must run a total of five simulations, described in Table \ref{tab:scenarios}. For each of our two science cases we run an
additional simulation with no planet in order to subtract out effects that are purely due to the shadow. Finally, we also run a control simulation without any shadow.
Figure \ref{fig:setup} shows the initial condition of each case described here.


\begin{table}
    \label{tab:scenarios}
    \begin{tabular}{ccccc}
        \hline
        \hline
        Name & $a$ & $b$ & $h$ & $m_p$ \\
        \texttt{no\_shadow} & - & - & 0 & $10^{-5}$ \\
        \texttt{narrow\_with} & 0.15 & 1.32 & 0.9 & $10^{-5}$ \\
        \texttt{narrow\_without} & 0.15 & 1.32 & 0.9 & 0 \\
        \texttt{wide\_with} & 0 & 0 & 0.9 & $10^{-5}$ \\
        \texttt{wide\_with} & 0 & 0 & 0.9 & 0 \\
        \hline
    \end{tabular}
    \caption{Simulation parameters. The planet mass of $10^{-5}$ is analogous to a planet three times as massive as Earth.}
\end{table}

\subsection{Simulation run}
\label{subsec:sim_run}

We ran each of the five simulations for 100 orbits ($t = 50\pi$) and output the simulation results as an HDF5 file every
quarter orbit ($\Delta t = \frac{\pi}{2}$). The end state of each simulation is shown in Figure \ref{fig:end}.

\begin{figure*}
    \includegraphics[width=1.0\textwidth]{figures/final_density.pdf}
    \caption{Final gas density anomaly maps for our five simulations. Note the patterns are dominated by
    the rings created by the shadow. The radial axis is scaled logarithmically.
    }
    \label{fig:end}
    \script{plot_final.py}
\end{figure*}

\subsection{Isolating planetary effects}
It is clear in Figure \ref{fig:end} that the major deviations in our shadowed disk are due to the shadow rather than the planet. Programically determining
the location of the planet's spiral peak is a non-trivial task. First, we determine the density anomaly $\Delta \rho = \rho - \rho_0$. In the regime where all perturbations
are small we expect that
\begin{equation}
    \Delta \rho_{\rm planet} = \Delta \rho - \Delta \rho_{\rm shadow}
    \label{eq:rho_planet}
\end{equation}

$\Delta \rho_{\rm shadow}$ is determined from our shadow-only simulations (\texttt{narrow\_without} and \texttt{wide\_without}). See Figures \ref{fig:sp_narrow}
\& \ref{fig:sp_wide} for this residual in the narrow and wide cases, respectively.

Now that we have a density anomaly map we need to determine the location of the spiral peak (in $\phi$) as a function of $r$. Absent of shadow perturbations
(i.e., in the \texttt{no\_shadow} case or if the subtraction in Equation \ref{eq:rho_planet} was perfect) this would be as simple as
finding the value of $\phi$ that maximizes $\rho(\phi,r)$ for any given $r$. However, this is not true in the general case as there are additional
effects that are only found in simulations containing both the planet and shadow. In order to ensure we don't find some other maximum we must constrain
the domain of our search along the $\phi$ axis. One constraint we can utilize is that $\phi_{\rm peak}(r=1) = \phi_{\rm planet}$ as that is where the spiral is
launched from. Since the spiral is a continuous function $\phi(r)$ (that is, its pitch angle is defined, see Section \ref{subsec:pitch}) we can assume that
$\phi(r_i) \approx 2\phi(r_{i-1}) - \phi(r_{i-2})$. Therefore, we can restrict our search for $\phi_{\rm peak}$ based on an initial guess. We restrict searches
to within 0.1 radians of this guess.

\begin{figure*}
    \script{plot_spirals_narrow.py}
    \includegraphics[width=1.0\textwidth]{figures/spirals_narrow.pdf}
    
    \caption{
        Spiral patterns observed after 99.25 orbits in the narrow shadow case. Shown is the density anomaly for every case.
        The third panel from the left shows the residuals after subtracting the first panel from the second. Plotted on the two
        rightmost panels are the locations identified as being at the peak of the spiral. Note that the color scale on the leftmost
        two panels has been scaled so that all panels can share the same colorbar.
    }
    \label{fig:sp_narrow}
\end{figure*}

\begin{figure*}
    \script{plot_spirals_wide.py}
    \includegraphics[width=1.0\textwidth]{figures/spirals_wide.pdf}
    \caption{ Same as Figure \ref{fig:sp_narrow} but for the wide shadow case.}
    \label{fig:sp_wide}
\end{figure*}

\section{Results}
\label{sec:results}

\subsection{Spiral Morphology}

\begin{figure}
    \begin{center}
        \includegraphics[width=0.5\textwidth]{figures/zeta_narrow.pdf}
        \script{plot_zeta_narrow.py}
        \caption{
            The pitch angle $\zeta$ for the narrow case. Far from the planet there is some discontinuous variability introduced by the shadow.
            The planet-only case fits well to the analytic approximation.
        }
        \label{fig:zeta_narrow}
    \end{center}
\end{figure}


\begin{figure}
    \begin{center}
        \includegraphics[width=0.5\textwidth]{figures/zeta_wide.pdf}
        \script{plot_zeta_wide.py}
        \caption{
            Same as Figure \ref{fig:zeta_narrow} but for the wide case. Note that the variations in the inset are much more continuous than in the
            narrow case, and have a much larger amplitude.
        }
        \label{fig:zeta_wide}
    \end{center}
\end{figure}

The shape of a spiral can be parameterized by its initial condition $\phi(r_0)$ and by its pitch angle $\zeta$. Pitch angle
is defined as
\begin{equation}
    \cot \zeta = \frac{d\phi}{d \ln r}
\end{equation}

according to \citet{zhu2022}. We use a central difference scheme to take the numerical derivative of $\phi_{\rm peak}$ with respect to $\ln r$.









\bibliography{syw,spirals}

\end{document}
